\documentclass[xcolor=pdftex,dvipsnames]{beamer}
\usepackage{beamerthemesplit,hyperref}
\usetheme{default}
\usecolortheme{dove}


\usepackage{listings,multirow,amssymb,amsmath,cancel,float,color}
\usepackage{pgfpages}

\graphicspath{}

%\setbeamertemplate{navigation symbols}{} 
%\setbeamertemplate{footline}[text line]{SUSY on the Lattice}

\mode<presentation>
\title[]{Cython for HPC}
\author[Niall Moran]{Niall Moran}
\date{July, 2016}
%\titlegraphic{\includegraphics[width=2.5cm]{NUIM-Logo-Main.pdf}}
\setbeamerfont{citation_font}{size=\tiny}
\setbeamerfont{smallerfont}{size=\small}

%\setbeameroption{show notes on second screen}
%\setbeameroption{show notes}
%\xdefinecolor{abricot}{named}{Apricot}

\newcommand{\ua}{\uparrow}
\newcommand{\da}{\downarrow}
\newcommand{\dg}{\dagger}
\newcommand{\nin}{\noindent}
\newcommand{\non}{\nonumber}
\newcommand{\pd}{\partial}
\newcommand{\bea}{\begin{eqnarray}}
\newcommand{\eea}{\end{eqnarray}}
\newcommand{\be}{\begin{equation}}
\newcommand{\ee}{\end{equation}}
\newcommand{\ba}{\begin{align}}
\newcommand{\ea}{\end{align}}
\newcommand{\braket}[2]{\langle #1|#2\rangle}
\newcommand{\ket}[1]{     |    \,    #1    \rangle}
\newcommand{\bra}[1]{  \langle #1  \,  |} 
\newcommand{\ZZ}{\mathbb{Z}}
\newcommand{\rmi}{\mathrm{i}}
\newcommand{\q}{{\bf q}}
\newcommand{\anb}{b^{\phantom\dagger}}
\newcommand{\crb}{b^\dagger}
\newcommand{\anc}{c^{\phantom\dagger}}
\newcommand{\crc}{c^\dagger}
\newcommand{\bi}{{\boldsymbol{i}}}
\newcommand{\bj}{{\boldsymbol{j}}}
\newcommand{\bk}{{\boldsymbol{k}}}
\newcommand{\bl}{{\boldsymbol{l}}}
\newcommand{\bx}{{\boldsymbol{x}}}
\newcommand{\bq}{{\boldsymbol{q}}}
\newcommand{\bp}{{\boldsymbol{p}}}
\newcommand{\bn}{{\boldsymbol{n}}}
\newcommand{\bs}[1]{ \boldsymbol{#1} }
\newcommand{\abs}[1]{|#1|}

\begin{document}


\frame{\titlepage
}


\frame{
	\frametitle{Overview}
	\begin{itemize}
		\item Motivation
		\item  {\color{Gray!30} Why (not) Python?}
		\item  {\color{Gray!30} Cython}
		\item  {\color{Gray!30} Examples}
	\end{itemize}
	\note{Overview slide}
}

\frame{
	\frametitle{Hall Effect}
  \begin{columns}
    \begin{column}{6cm}
      \begin{itemize}
        \item Edwin Hall (date)
        \item Magnetic field sensor
      \end{itemize}
      image of experimental setup\\
      image of joystick
    \end{column}
    \begin{column}{6cm}
      plot of conductivity with field
    \end{column}
  \end{columns}
	\note{Overview slide}
}

\frame{
	\frametitle{Quantum Hall Effect}
  \begin{columns}
    \begin{column}{6cm}
      \begin{itemize}
        \item Pffifer et. al. (date)
        \item Stronger field, purer samples
        \item Used for ohm standard
        \item Single particle QM
      \end{itemize}
      experimental photo\\
      image of joystick
    \end{column}
    \begin{column}{6cm}
      plot of conductivity with field showing plateaus at
      integer filling
    \end{column}
  \end{columns}
	\note{Overview slide}
}

\frame{
	\frametitle{Fractional Quantum Hall Effect}
  \begin{columns}
    \begin{column}{6cm}
      \begin{itemize}
        \item Pffifer et. al. (date)
        \item Stronger field, purer samples again
        \item Unexplained exotic phases
        \item Still of interest
      \end{itemize}
      experimental photo\\
    \end{column}
    \begin{column}{6cm}
      plot of conductivity with field showing plateaus at
      fractional filling and experimental plot.
    \end{column}
  \end{columns}
	\note{Overview slide}
}

\frame{
  \frametitle{Model}
  \begin{itemize}
    \item Effective one dimensional model
    \item State is made up of (complex) weight in each configuration
  \end{itemize}
  picture of boxes with balls\\

  equation showing exponential scaling\\
}

\frame{
  \frametitle{Hopeless scaling}
  \begin{columns}
    \begin{column}{6cm}
      image of laptop\\
      10-12 electrons
    \end{column}
    \begin{column}{6cm}
      image of supercomputer\\
      16-18 electrons
    \end{column}
    \begin{column}{6cm}
    \end{column}
  \end{columns}
}


\frame{
  \frametitle{Why use Python}
  \begin{itemize}
    \item Fast development, debugging
    \item Amount of packages available
    \item Plotting
    \item Flexibility
    \item Good glue
  \end{itemize}
}


\frame{
  \frametitle{Why not use Python}
  \begin{itemize}
    \item Not always available
    \item Slow
    \item GIL
    \item Resource usage
  \end{itemize}
}


\frame{
  \frametitle{Cython}
  \begin{itemize}
    \item Mix C and python
    \item Best of both
    \item Good for wrapping existing C code
    \item Two language problem
  \end{itemize}
}


\frame{
  \frametitle{Usage}
  Figure of pipeline
}


\frame{
  \frametitle{Example - setup.py}
  Show setup.py and commands
}



\frame{
  \frametitle{Example - jupyter}
  Show cython magic in the browser
}


\frame{
  \frametitle{Benchmark}
  Show example and compare to python
}


\frame{
  \frametitle{Optimising}
  Show example and compare to python
}


\frame{
  \frametitle{Further reading}
  \usebeamerfont{smallerfont}
  \begin{itemize}
    \item D. S. Seljebotn, \textit{``Fast Numerical Computation with Cython''}, SciPy 2009.{\usebeamerfont{citation_font} (\url{http://conference.scipy.org/proceedings/SciPy2009/paper\_2/full\_text.pdf})}
  \end{itemize}
}


\end{document}
